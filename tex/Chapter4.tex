%THIS IS HOW TO PLACE FIGURES EXACTLY WHERE YOU WANT THEM
%\begin{figure}[h]
%\begin{center}
%\includegraphics[scale=1]{graphics/diskPackingTheoremExample.pdf}
%\end{center} 
%\caption{This example represents a disk arrangement transformed to and from its corresponding graph 
%$G_2$}
%\label{fig:DiskArrangement-1}
%\end{figure}
%THIS IS HOW TO PLACE FIGURES EXACTLY WHERE YOU WANT THEM
%FOR FURTHER REFERENCE, READ http://en.wikibooks.org/wiki/LaTeX/Floats,_Figures_and_Captions#Keeping_floats_in_their_place
\chapter{Numerical Solution, Conclusion and Future Work}
Since we have done lot of good work, now it is the time to check the implementation.
We will provide examples which will give better understanding for our problem.
{Numberical Solutions using implementation}
\section{Examples}
\subsection{EXAMPLE 1}
\begin{itemize}
  \item Ticker: \textbf{MWI Veterinary Supply Inc}
  \item  D : 05/16/2014
  \item  T : 60 seconds
\end{itemize}
\textbf{Stock Class }\\
We are using NASD. We download information from following website. Figure 1.1 shows stock prices vs. time in seconds.
\begin{figure}[h]
\begin{center}
\includegraphics[scale=0.40]{MWIV_stock_price.png}
\end{center}
\caption{Stock Prices vs. Time}
\label{fig:Stock Price}
\end{figure}
\newpage
Now we have stock prices for MWI Veterinary. We will use Floren Zmirou estimator to see the volatility of stock prices.\\
\textbf{Floren Zmirou Class}
\begin{center}
\begin{equation}\label{florenZmirouEquation}
S_n(x) = \frac{\Sigma_{i=1}^{n} 1_{\{|S_{t_i}-x|<h_n\}} n (S_{t_i+1}-S_{t_i})^2}{\Sigma_{i=1}^{n} 1_{\{|S_{t_i}-x|<h_n\}}}
\end{equation}
\end{center}
\begin{tabular}{|1|c|r|}
\hline
Usable Grid Points  &  Estimated Sigma Zmirou  & Number of Points\\
\hline
141.842890874       &           1897.69862662  &              50\\
\hline
144.17445437        &          290.806107556   &            108\\
\hline
143.008672622       &           464.127160557  &              60\\
\hline
\end{tabular}
\begin{figure}[h]
\begin{center}
\includegraphics[scale = 0.40]{MWIV_FZ_stddev_estimation.png}
\end{center}
\caption{Stock Prices vs.Floren Zmirou Standard Deviation Estimation}
Figure 1.2 shows volatility vs. stock prices. There are Floren Zmirou's estimated sigma values for 
usable grid points and number of points in each usable grid point. Next we used Cubic spline to connect
Floren Zmirou's sigma points. 
\label{fig:Floren Zmirou Estimation}
\end{figure}
\newpage
\textbf{Cublic Spline}
\begin{figure}[h]
\begin{center}
\includegraphics[scale=0.50]{MWIV_variance_spline.png}
\end{center}
\caption{Floren Zmirou Standard Deviation Estimation vs. Variance Cubic Spline}
\label{fig:Cubic Spline}
\begin{center}
\includegraphics[scale=0.45]{MWIV_stddev_spline.png}
\end{center}
\caption{Floren Zmirou Standard Deviation Estimation vs.Standard Deviation Cubic Spline}
\label{fig:Cubic Spline}
\end{figure}
\\
In above example,  
%%%%%%%%%%%%%%%%%%%%%%%%%%%%%%%%%%%%%%%%%%%%%%%%%%%%%%%%%%%%%%%%%%%%%%%%%%%%%%%%%%%%%%%
\subsection{EXAMPLE 2}
\begin{itemize}
  \item Ticker:\textbf{ GOOGLE Inc.}
  \item  D : 05/16/2014
  \item  T : 60 seconds
\end{itemize}
\textbf{Stock Class: }\\
\begin{figure}[h]
\begin{center}
\includegraphics[scale=0.40]{GOOG_stock_price.png}
\end{center}
\caption{Stock Prices vs. Time}
\label{fig:Stock Price}
\end{figure}
\\
\textbf{Floren Zmirou Class}\\\\
\begin{tabular}{|1|c|r|}
\hline
Usable Grid Points &   Estimated Sigma Zmirou  & Number of Points\\
\hline
547.289611925      &            267.623605573  &              64\\
\hline
549.612686541      &            517.868135963  &             143\\
\hline
551.935761158      &           76.0733825073   &              17\\
\hline
544.966537308      &               1890.46832  &               72\\
\hline
\end{tabular}
\begin{figure}[h]
\begin{center}
\includegraphics[scale=0.50]{GOOG_FZ_stddev_estimation.png}
\end{center}
\caption{Floren Zmirou Standard Deviation Estimation vs. Stock Prices}
\label{fig:Stock Price}
\end{figure}
\newpage
\textbf{Cublic Spline}
\begin{figure}[h]
\begin{center}
\includegraphics[scale=0.50]{GOOG_variance_spline.png}
\end{center}
\caption{Floren Zmirou Standard Deviation Estimation vs. Variance Cubic Spline}
\label{fig:Cubic Spline}
\begin{center}
\includegraphics[scale=0.45]{GOOG_stddev_spline.png}
\end{center}
\caption{Floren Zmirou Standard Deviation Estimation vs.Standard Deviation Cubic Spline}
\label{fig:Cubic Spline}
\end{figure}
\\
In above example,  
%%%%%%%%%%%%%%%%%%%%%%%%%%%%%%%%%%%%%%%%%%%%%%%%%%%%%%%%%%%%%%%%%%%%%%%%%%%%%%%%%%%%%%%
\subsection{EXAMPLE 3}
\begin{itemize}
  \item Ticker: \textbf{APPLE Inc.}
  \item  D : 05/21/2014
  \item  T : 60 seconds
\end{itemize}
\textbf{Stock Class}
\begin{figure}[h]
\begin{center}
\includegraphics[scale=0.40]{AAPL_stock_price.png}
\end{center}
\caption{Stock Prices vs. Time}
\label{fig:Stock Price}
\end{figure}
\\
\textbf{Floren Zmirou Estimation}\\
\begin{center}
\begin{equation}\label{florenZmirouEquation}
S_n(x) = \frac{\Sigma_{i=1}^{n} 1_{\{|S_{t_i}-x|<h_n\}} n (S_{t_i+1}-S_{t_i})^2}{\Sigma_{i=1}^{n} 1_{\{|S_{t_i}-x|<h_n\}}}
\end{equation}
\end{center}
\\
\begin{tabular}{|1|c|r|}
\hline
Usable Grid Points &   Estimated Sigma Zmirou &  Number of Points\\
\hline
602.871457276      &           138.351149247      &           42\\
\hline
603.874371827      &            245.251175157     &           125\\
\hline
604.877286378      &            102.97102087      &          104\\
\hline
\end{tabular}
\begin{figure}[h]
\begin{center}
\includegraphics[scale=0.50]{AAPL_FZ_stddev_estimation.png}
\end{center}
\caption{Floren Zmirou Standard Deviation Estimation vs. Stock Prices}
\label{fig:Stock Price}
\end{figure}
\newpage
\textbf{Cublic Spline}
\begin{center}
\includegraphics[scale=0.50]{AAPL_variance_spline.png}
\end{center}
\caption{Floren Zmirou Standard Deviation Estimation vs. Variance Cubic Spline}
\label{fig:Cubic Spline}
\begin{center}
\includegraphics[scale=0.45]{AAPL_stddev_spline.png}
\end{center}
\caption{Floren Zmirou Standard Deviation Estimation vs.Standard Deviation Cubic Spline}
\label{fig:Cubic Spline}
\\
In above example,  
%%%%%%%%%%%%%%%%%%%%%%%%%%%%%%%%%%%%%%%%%%%%%%%%%%%%%%%%%%%%%%%%%%%%%%%%%%%%%%%%%%%%%%%
\subsection{EXAMPLE 4}
\begin{itemize}
  \item Ticker: \textbf{GROUPON Inc.}
  \item  D : 05/21/2014
  \item  T : 60 seconds
\end{itemize}
\textbf{Stock Class}
\begin{figure}[h]
\begin{center}
\includegraphics[scale=0.40]{GRPN_stock_price.png}
\end{center}
\caption{Stock Prices vs. Time}
\label{fig:Stock Price}
\end{figure}
\\
\textbf{Floren Zmirou Estimation}\\
\begin{center}
\begin{equation}\label{florenZmirouEquation}
S_n(x) = \frac{\Sigma_{i=1}^{n} 1_{\{|S_{t_i}-x|<h_n\}} n (S_{t_i+1}-S_{t_i})^2}{\Sigma_{i=1}^{n} 1_{\{|S_{t_i}-x|<h_n\}}}
\end{equation}
\end{center}

\begin{tabular}{|1|c|r|}
\hline
Usable Grid Points  &  Estimated Sigma Zmirou &  Number of Points\\
\hline
6.01159662403       &        0.0106001479673  &             25\\
\hline
6.0947898721        &        0.00331796023881 &              38\\
\hline
6.17798312017       &       0.000229293586847 &              155\\
\hline
6.26117636824       &      7.51642424675e-05  &              86\\
\hline
\end{tabular}
\begin{figure}[h]
\begin{center}
\includegraphics[scale=0.50]{GRPN_FZ_stddev_estimation.png}
\end{center}
\caption{Floren Zmirou Standard Deviation Estimation vs. Stock Prices}
\label{fig:Stock Price}
\end{figure}
\newpage
\textbf{Cublic Spline}
\begin{figure}[h]
\begin{center}
\includegraphics[scale=0.50]{GRPN_variance_spline.png}
\end{center}
\caption{Floren Zmirou Standard Deviation Estimation vs. Variance Cubic Spline}
\label{fig:Cubic Spline}
\begin{center}
\includegraphics[scale=0.45]{GRPN_stddev_spline.png}
\end{center}
\caption{Floren Zmirou Standard Deviation Estimation vs.Standard Deviation Cubic Spline}
\label{fig:Cubic Spline}
\end{figure}
\\
\newpage
\section{Future Work}
\begin{itemize}
  \item Still need to know tail of the volatility function.
  \item Need to extrapolate the volatility with either Comparison Theorem Method or Reproducing Kernal Hilbert Spaces.
  \item Need to know themorem 0.1.12 equation(3)
  \begin{equation}
  \int_\alpha^\infty \frac{x}{\sigma^2(x)}dx < \infty
   \end{equation} 
   for all $\alpha > 0$ is finite or infinite.
  \item Determine from intergral if there is bubble or not.
\end{itemize}
