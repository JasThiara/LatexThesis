\section{Introduction}
Today's economy, financial asset bubbles are exciting and hot topic. In most recent market news, we have read or seen big changes in Gold prices. Everyone is interested to know what will happen in the future. How we can able to detect or estimate the future changes of any asset ( stock, gold, housing, commodity)? How quickly asset price will jump? These are the question which we will consider in this study. We will study how to determine whether any asset is experiencing a price bubble in real time. How we will detect asset bubbles in real time?\\

 First of all we will introduce Stochastic Differential Equations. SDE are being used in various fields for example biology, physics, mathematics and of course finance. In finance, SDE is used to model asset price included with Brownian motion. Here we will use constant parameters drift and diffusion coeficients. With these constants , we will use euler muruyama method to model asset price. \\
 
 Second we will introduce Floren Zmirou's nonparametric estimator which is based on local time of the diffusion process on compact domain. we only can estimate $\sigma(x)$ on points visited by the process. So we cant determine the tail of the volatility function to check whether the we have bubble or not. \\
 
 Third, now we want to check the tail, so we will use RKHS (Reproducing Kernal Hilbert Space)  method which will allow us to construct an interpolationg function that extends the nonparametric estimator from the observation interval to the entire real line.\\
 
 With the help of all these steps, we can able to estimate the fuction which will detect whether asset price has bubble or not.\\
 
 


