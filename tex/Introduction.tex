\section{Introduction}
Chapter 1\\\\
Question 1.1 What problem are you trying to solve?  Why is it important? What is your goal?
Answer: Today's economy, financial asset bubbles are exciting and hot topic. 
In most recent market news, we have read or seen big changes in Gold prices. 
Everyone is interested to know what will happen in the future. 
How we can able to detect or estimate the future changes of any asset ( stock, gold, housing, commodity)? 
How quickly asset price will jump? These are the question which we will consider in this study. 
We will study how to determine whether any asset is experiencing a price bubble in real time. 
How we will detect asset bubbles in real time?\\
Our problem will be deciding if an asset price is experiencing a price bubble in finite and infinite time period.\\
We will able to determine the volatility of asset price.Using some helpful techniques to determine asset bubble in real time will help finanial corporations, banks,and money marrkets.\\
They can lower their money damaging risks by using our methodology.

According to `` There is a bubble'' paper paragraph 2, \\
``indeed 2009 the federal reserve chairman Ben bernanke said in congress testimony[1]\\
``It is extraordinary difficult in real time to know if an asset price is appropriate or not''.\\
Our goal is to estimate stock price volatility by Floren Zmirou estimator and then we will extrapolate the volatility tale in order to check the integral\\.
wheather the integral is finite or infinite. \\
 The process for bubble detection depends on a mathematical analysis that determines when an asset is undergoing speculative pricing\\
 i.e its market price is greater than its fundamental price.\\
 The difference between market and fundamential price, is a price bubble 
 First of all we will introduce Stochastic Differential Equations. 
 SDE are being used in various fields for example biology, physics, mathematics and of course finance. 
 In finance, SDE is used to model asset price included with Brownian motion. 
 Here we will use constant parameters drift and diffusion coeficients. 
 With these constants , we will use euler muruyama method to model asset price. \\
 
 
 


