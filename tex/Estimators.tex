
\chapter{Implementation of an Asset Bubble Problem}
\section{Implementation}
We used sage python to solve this problem. We orgnize our data and process in six classes. Each class is designed to store the information about Stock, Floren Zmirou
AssetBubble, AssetBubbleDetection, Approximation, and Run.
\subsection{Stock Class}
\paragraph{Methods of Stock Class}
We obtain stock data by two types of methods. First, we will get it through Google Finance and second, we will get it from CSV file.
\begin{enumerate}
  \item \verb! IsNumber()!: Input for this function is rowValue. This function determine if rows of stock data is a numerical string or not.
  \item \verb! GetGoogleData()!: Input for this function are Ticker,days and period. This function obtains data for any stock from Google
  finance.
  \begin{algorithm}
  \caption{GetGoogleData ()}
  \bigskip
  \textbf{Inputs}: Ticker, days, period\\
  \textbf{Steps}
  \begin{algorithmic}[1]
  \If {the length of the Ticker is less than equal to 3} 
  \State{exchange it with New York Stock Exchange (NYSE)}.
  \Else 
  \State {exchange it with Natinal Association of Securities Dealers (NASD)}
  \EndIf\\
  We initialize current time in integer.\\
  We will open Google Finance link.\\
  We initialize dataList to read each line from opened link.\\
  We initialize tickerData to be the list of array of dataList.\\
  We will put stockPrices in list.\\
  We initialize  minuteData.\\
  \For {minuteData in tickerData}
  \State We initialize datum and put split minuteData in commas in datum.
  \State We will append datum row one in float and store it in stockPrice list.\\
  \Return {stockPrices}
  \end{algorithmic}
  \end{algorithm}
  \item \verb! GetStockPrices()!: Input for this function is csv filename. This function obtains stock prices which are stored
  in csv file by filename. It will read the yahoo API based minute to minute data.
  \begin{algorithm}
  \caption{GetStockPrices ()}
  \bigskip
  \textbf{Inputs}: filename\\
  \textbf{Steps}
  \begin{algorithmic}[1]
  \item We saved stock Prices downloaded from yahoo in csv filename.
  \item We initialize cr which will open and read the csv filename.
  \item We skip the header.
  \item Define c1 to be the empty list.
  \For {row in cr}
  \If {the second row is numberical string}
  \State {we add second row to c1.}\\
  \Return{c1}
  \end{algorithmic}
  \end{algorithm}
  \item \verb! __init__()!: Input for this function is keyward Usage.This function will analyzie wheather csv filename is used or Ticker parameter 
  for Google finance.
  \begin{algorithm}
  \caption{--init--()}
  \bigskip
  \textbf{Inputs}: Keyword Usage \\
  \textbf{Steps}
  \begin{algorithmic}[1]
  \If {filename is in keyward Usage}
  \State We  will obtain Stock Prices from GetStockPrices in list of filename
  \Elif{Ticker is used in keyward Usage}
  \State{We will obtain Stock Prices from GetGoogleData in a list of Ticker}
  \Else{We will give exception of bad parameters}
  \end{algorithmic}
  \end{algorithm}
\subsection{Floren Zmirou Class}
\paragraph{Process to solve Floren Zmirou}
Now we have Stock Prices from Stock class. We will obtain list of sigma values from list of Stock Prices.We will explain this class in following algorithms:\\\\
T = 60*n where T is the minute to minute time period [0,T].\\
Now we will derive hn, x-step\_size, and x values which will be used in Floren Zmirou Estimator.
  \item \verb! (Derive_hn)!: Input for this function is Stock Prices which we obtained from Stock class. 
  This function will derive hn which will be used in Floren Zmirou Estimator.
  \begin{algorithm}
  \caption{Derive hn ()}
  \bigskip
  \textbf{Inputs}: Stock Prices S\\
  \textbf{Steps}
  \begin{algorithmic}[1]
  \item n = length of Stock Prices S.
  \item hn = 1.0/n**(1.0/3.0)
  \item return hn
  \end{algorithmic}
  \end{algorithm}
  \item \verb! (x_step_size)!: Input for this function is Stock Prices which we obtained from Stock class. 
  This function will derive x\_step\_size which will be used  to create step size to generate x.
  \begin{algorithm}
  \caption{x step size ()}
  \bigskip
  \textbf{Inputs}: Stock Prices S\\
  \textbf{Steps}
  \begin{algorithmic}[1]
  \item We obtained hn from Derive hn function.
  \item Double hn = 2 **hn
  \item Difference = max S-min S
  \item x\_hn = Difference * Double hn
  \item Return x\_hn
  \end{algorithmic}
  \end{algorithm}
  \end{algorithmic}
\end{enumerate}
