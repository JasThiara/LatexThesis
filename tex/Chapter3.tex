\chapter{Implementation of an Asset Bubble}
In this chapter. we will provide valueable information about the methods used in our problem. 
We used sage python to solve this problem. We orgnize data and process in four classes. 
Each class is designed to store the information about Stock, Floren Zmirou
,Cubic Spline,and Run.
\section{Stock} 
Let's consider following questions: 
\begin{itemize}
  \item What kind of Data will be used?
  \item How can someone obtain accurate historic data information?
  \item Will it be Mintue to mintue data information?
  \item How we will store and use data information?
  \item Can we get data information for any stock symbol in market?
\end{itemize} 
Following Stock Class which will provide answers to all the above questions.
\subsection{Stock Class}
We used Google Finance and CSV file from Yahoo Finance methods to download stock data information.
There are GetGoogleData, GetStockPrice and init are the four algorithms used in Stock Class which will decribe the process.
\begin{itemize}
  \item S Stock Price.
  \item Smax Maximum Stock Price. 
  \item Smin Minimum Stock Price.
  \item Row of Stock prices is numberical string.
  \item Ticker is Ticker symbol name.
  \item D Number of days.
  \item T time in seconds in intergers.
\end{itemize}
We will read, download and save stock data by following methods.
\subsubsection {Google Finance Stock Prices}
\begin{enumerate}
\item  Get Google Data: This function obtains mintue to mintue stock prices in numberical string for any ticker from Google Finance and save in array. 
\begin{algorithm}
\caption{Get Google Data}
\bigskip
\textbf{INPUT}: Ticker, D ,T\\
\textbf{OUTPUT}: $S = (S_{t_1}\ldots S_{t_n})$ in [0,T] where $t_i = \frac{i}{n}T$.
\begin{algorithmic}[1]
  \If {Ticker length $\leq$ 3}
  \State{exchange Ticker with New York Stock Exchange (NYSE)}.
  \Else
  \State {exchange Ticker with Natinal Association of Securities Dealers (NASD)}
  \EndIf
  \item Open Google Finance link.
  \item DataList = reads each line from Google Finance link.
  \item TickerData = list of array of DataList.
  \item Put stock Prices in a list.
  \For {MinuteData in TickerData}
  \State {Seperate MintueData with commas and store in a list S.}
  \State {Append S as float and save it.}\\
  \Return {S = Stock prices}
\end{algorithmic}
\end{algorithm}
\subsubsection{Yahoo Finance Stock Prices}
\item  Get Yahoo Data : This function obtains the stock prices which are downloaded from yahoo finance and stored
in csv file by filename. It will provide the yahoo API based minute to minute data.
\begin{algorithm}
\caption{Get Yahoo Data}
\bigskip
\textbf{INPUT}: Filename\\
\textbf{OUTPUT}: S from CSV file.
\begin{algorithmic}[1]
  \item Open Yahoo Finace link.
  \item Open and read the csv Filename as universal.
  \item Skip the header.
  \item Create a empty list.
  \For {row in csv Filename}
  \If {the second row is numberical string}
  \State {Add second row to empty list}
  \item Name this row S.\\
  \Return{S}
\end{algorithmic}
\end{algorithm}
\newpage
\subsubsection{Choosing either Google Ticker Or Yahoo CSV file}
\item  Class Constructor: This function will analyzie wheather to use csv filename for Yahoo or Ticker 
for Google finance.
\begin{algorithm}
\caption{--init--}
\bigskip
\textbf{INPUT}: User Keyword  \\
\textbf{OUTPUT}: S 
\begin{algorithmic}[1]
  \If {Filename is in keyward Usage}
  \State {Obtain Stock Prices from Get Stock Prices.}
  \Elif {Ticker is used in keyward Usage}
  \State {Obtain Stock Prices from Get Google Data.}
  \Else { Bad parameters}\\
  \Return{S}
\end{algorithmic}
\end{algorithm}
\subsection {Example}
\newpage
\section{Floren Zmirou}
This class will implement the following equation:
\begin{equation}\label{florenZmirouEquation}
S_n(x) = \frac{\Sigma_{i=1}^{n} 1_{\{|S_{t_i}-x|<h_n\}} n (S_{t_i+1}-S_{t_i})^2}{\Sigma_{i=1}^{n} 1_{\{|S_{t_i}-x|<h_n\}}}
\end{equation}
Now we have Stock Prices from Stock Class. Our goal is to get $\sigma(x)$ values from list of Stock Prices S.
We will import sage and stock class to Floren Zmirou class. In order to use Floren Zmirou Estimator, we will need following:\\
\begin{itemize}
  \item n = Number of stock prices.
  \item T = Time in seconds
  \item hn = $1/(n)^(1/3)$
  \item x-step\-size
  \item x values
\end{itemize}
\begin{itemize}
  \item We will implement hn, x-step\-size, and x values so we can use the terms in Floren zmirou estimator equation (1.1).
\end{itemize}
\subsection{Floren Zmirou Class}
\begin{algorithm}
\caption{Derive hn}
\bigskip
\textbf{INPUT}: Stock Prices S\\
\textbf{OUTPUT}: hn
\begin{algorithmic}[1]
  \item n = length of Stock Prices S.
  \item hn = 1.0/n**(1.0/3.0)\\
  \Return {hn}
\end{algorithmic}
\end{algorithm}
\item (x step size): This function will be used  to create step size to generate x.
\begin{algorithm}
\caption{x\_hn = x step size}
\bigskip
\textbf{INPUT}: Stock Prices S\\
\textbf{OUTPUT}: x\_hn
\begin{algorithmic}[1]
  \item Double hn = 2 **hn
  \item Difference = max S-min S
  \item x\_hn = Difference * Double hn\\
  \Return {x\_hn}
\end{algorithmic}
\end{algorithm}
\begin{algorithm}
\caption { x Values}
\bigskip
\textbf{INPUT}: Stock Prices S\\
\textbf{OUTPUT}: x
\begin{algorithmic}[1]
  \item half h\_n = x\_hn /2.0
  \item x = empty list.
  \item Append x with Smin + half h\_n
  \item FirstElement = first element of x's list.
  \While {FirstElement < Smax}
  \State{FirstElement = FirstElement + x\_hn}
  \EndWhile
  \item Append FirstElement into x's list.\\
  \Return {x}
\end{algorithmic}
\end{algorithm}
\newpage
\begin{itemize}
  \item Now we will implement floren Zmirou Estimator. Floren Zmirou has Sublocal time, Local time, volatility Estimator and Indicator function.
\end{itemize}
\item Sublocal Time:This function will give $L_{T}^n(x) = (\frac{T}{2nhn}) \sum_{i =1}^ n 1_{\left\vert S_{t_i} - x \right\vert < h_n}$
\begin{algorithm}
\caption{Sublocal Time}
\bigskip
\textbf{INPUT}: T, S, x, n, hn.\\
\textbf{OUTPUT}: $L_{T}^n(x) = (\frac{T}{2nhn}) \sum_{i =1}^ n 1_{\left\vert S_{t_i} - x\right\vert }$
\begin{algorithmic}[1]
  \item sum = 0.0
  \item scalar = T/(2.0*n*hn)
  \For {i in range of the length of Stock Prices}
  \State{Sti = i th element of the list of Stock prices}
  \State{absoluteValue = abs(Sti-x)}
  \State{indicatorValue to pass through indicator function}
  \If{absoluteValue is less than hn}
  \State {sum = sum+indicatorValue}\\
  \Return{scalar*sum}
\end{algorithmic}
\end{algorithm}
\item Local Time: $L_{T}^n(x) = (\frac{T}{2nh_n}) \sum_{i =1}^ n 1_{\left\vert S_{t_i} - x\right\vert  < hn}*n(S(t(i+1))-S(t(i))^2$
\begin{algorithm}
\caption{Local Time}
\bigskip
\textbf{INPUT}: T, S, x, n, hn.\\
\textbf{OUTPUT}:
\begin{algorithmic}[1]
  \item sum = 0.0
  \item scalar = T/(2.0*n*hn)
  \For{ i in range of the length of Stock prices - 1}
  \State{Sti = ith element of the list of Stock prices}
  \State{Stj = jth element of the list of Stock prices}
  \State{absoluteValue = abs(Sti-x)}
  \State{Difference = (Stj-Sti)**2}
  \If{ absoluteValue is less than hn}
  \State{sum = sum+indicatorValue*n*Difference}\\
  \Return{scalar*sum}
\end{algorithmic}
\end{algorithm}
\item Volatility Estimator Sn(x): Volatility Estimator is $\sigma^2(x)$. This function will give Local Time$/$Sublocal Time.
\begin{algorithm}
\caption{Volatility Estimator}
\bigskip
\textbf{INPUT}: T, S, x, n, hn.\\
\textbf{OUTPUT}:
\begin{algorithmic}[1]
  \item Ratio =  Local Time / Sublocal Time.\\
  \Return {Ratio}
\end{algorithmic}
\end{algorithm}
\newpage
\begin{itemize}
  \item We have $\sigma^2(x)$ for Stock Prices. Now we want to see how many stock prices are in each grid point.
If there are less than 0 or 1 percent of stock prices in each grid then we will exclude that grid point from
our calculation process.
\end{itemize}
\item (DoGridAnalysis): This function give us the list of usable grid points and for each usable grid point, the list of Stock prices.
\begin{algorithm}
\caption{DoGridAnalysis}
\bigskip
\textbf{INPUT}: T,S,x,n,hn.\\
\textbf{OUTPUT}:
\begin{algorithmic}[1]
  \item x = useableGridPoints.
  \item d = empty dictionary
  \item Si =  length of stock prices.
  \For{grid Points in x}
  \State{create a dictionary where the keys are the grid points and the value is a list for each grid point}
  \For{stockPrice in S}
  \If{|grid Points - stockPriceCount|$\leq$ hn}
  \State{Add x value to corresponding Si}
  \For{grid Points in x}
  \If{if the list of data points corresponding to x values than Y percent of total grid points}
  \State{add it to the list of usable grid points}
  \State L = dictionary with key values of grid points.
  \State N = Length of L in float.
  \State {Percent of Stock Prices = N/S}
  \If {Percent of Stock prices < Y}
  \State{Remove grid points from x}
  \State{Delete L}\\
  \Return{x = usable grid points and d }
\end{algorithmic}
\end{algorithm}
\begin{algorithm}
\caption{Get Grid Variance}
\bigskip
\textbf{INPUT}: S,x\_hn, x values\\
\textbf{OUTPUT}: Floren Zmirou estimation over usable grid points.
\begin{algorithmic}[1]
  \item Points = Empty list.
  \item half x\_hn = x step size*(Stock Prices)/2.0
  \For{x in usable grid points}
  \State{y =Sn(x)**2}
  \State{Append (x,y)}\\
  \Return{(x,y)}
\end{algorithmic}
\end{algorithm}
\begin{algorithm}
\caption{GetGridInverseStandardDeviation}
\bigskip
\textbf{INPUT}:S,x\_hn,x values.\\
\textbf{OUTPUT}: Standard Deviation of usable grid points.
\begin{algorithmic}[1]
  \item Points = Empty list
  \For{x in usable grid points}
  \State{y = 1/sqrt(Sn(x))**2}
  \State{Append (x,y) in Points's list}\\
  \Return{(x,y)}
\end{algorithmic}
\end{algorithm}
\begin{algorithm}
\caption{GetGridSigma}
\bigskip
\textbf{INPUT}:S,x\_hn,x values.\\
\textbf{OUTPUT}: Standard Deviation of usable grid points.
\begin{algorithmic}[1]
  \item Points = Empty list
  \For{x in usable grid points}
  \State{y = Sn(x)}
  \State{Append (x,y) in Points's list}\\
  \Return{(x,y)}
\end{algorithmic}
\end{algorithm}
\section{ Natural Cubic Spline}- To construct the cubic spline interpolant $S$ for the function $f$, defined
at the numbers $x_0<x_1<\ldots<x_n$ satisfying $S''(x_0) = S''(x_n) = 0:$\\
(Natural Cubic Spline)
  \begin{algorithm}
  \caption{Natural Cubic Spline}
  \bigskip
  \textbf{INPUT}: $n, x_0,x_1,\ldots,x_n; a_0 = f(x_0), a_1 = f(x1),\ldots, a_n = f(x_n)$\\
  \textbf{OUTPUT}: $a_j, b_j,c_j,d_j$ for $j = 0,1,\ldots,n-1$.\\
  (note: $S_(x) = S_j(x) = a_j+b_j(x-x_j)+c_j(x-x_j)^2+d_j(x-x_j)^3$ for $x_j\leq x \leq x_{j+1}$)
  \begin{algorithmic}[1]
  \For{$i = 0,1,$\ldots$,n-1$}
  \State{$h_i = x_{i+1}-x_i$}
  \EndFor
  \For{$i = 1,2,\ldots,n-1$}:
  \State{$\alpha_i = \frac{3}{h_i}(a_{i+1}-a_i)-\frac{3}{h_{i-1}}(a_i-a_{i-1})$}
  \EndFor
  \item Set $l_0 = 1;$
  \item $\mu = 0;$
  \item $z_0 = 0.$
  \For{$i = 1,2,\ldots,n-1$}
  \State{$l_1 = 2(x_{i+1}-x_{i-1})-h_{i-1}\mu_{i-1}$};
  \item $\mu_1 = \frac{h_1}{l_1};$
  \item $z_1 = \frac{(\alpha_1 - h_{i-1}z_{i-1})}{l_1}.$
  \item Set $l_n = 1;$
  \item $z_n = 0;$
  \item $c_n = 0.$
  \For{$j = n-1,n-2,\ldots,0$}
  \State{$ c_j = z_j-\mu_jc_{j+1};$}
  \State{$b_j = (a_{j+1}-a_j)/h_j -h_j(c_{j+1}+2c_j)/3;$}
  \State{$d_j = (c_{j+1}-c_j)/(3h_j)$}
  \item Return $(a_j,b_j,c_j,d_j for j = 0,1,\ldots,n-1)$;
  \end{algorithmic}
\end{algorithm}
\end{enumerate}
\section{Run the Program}
