\chapter{Implementation of an Asset Bubble}
In this chapter. we will provide detail information about the methods used in our problem. 
We used sage python to solve this problem. We orgnize our data and process in six classes. 
Each class is designed to store the information about Stock, Floren Zmirou
,Cubic Spline,and Run.
\section{Stock} 
Let's consider these following questions: 
\begin{itemize}
  \item What kind of Data will be used?
  \item How someone can obtain accurate historic data information?
  \item Will it be Mintue to mintue data information?
  \item How we will store and use data information?
  \item Can we get data information for any stock symbol in market?
\end{itemize} 
Stock Class which will provide answers to all the above questions.
\subsection{Stock Class}
There are two types of methods first Google Finance and second CSV file from Yahoo Finance.
IsNumber,GetGoogleData, GetStockPrice,and init are the four algorithms used in Stock Class.
\begin{itemize}
  \item S Stock Price.
  \item Smax Maximum Stock Price. 
  \item Smin Minimum Stock Price.
  \item Row of Stock prices is numberical string.
  \item Ticker symbol name.
  \item D Number of days.
  \item T time in seconds interger.
\end{itemize}
\subsubsection {Google Finance Stock Prices}
\begin{enumerate}
\item  Get Google Data: This function obtains mintue to mintue stock prices in numberical string for any ticker from Google Finance and save in array. 
\begin{algorithm}
\caption{Get Google Data}
\bigskip
\textbf{INPUT}: Ticker,D,T\\
\textbf{OUTPUT}:$S = (S_{t_1}\ldots S_{t_n})$ in [0,T] where $t_i = \frac{i}{n}T$.
\begin{algorithmic}[1]
  \If {Ticker length $\leq$ 3}
  \State{exchange Ticker with New York Stock Exchange (NYSE)}.
  \Else
  \State {exchange Ticker with Natinal Association of Securities Dealers (NASD)}
  \EndIf
  \item Open Google Finance link.
  \item DataList = reads each line from Google Finance link.
  \item TickerData = list of array of DataList.
  \item Put stock Prices in a list.
  \For {MinuteData in TickerData}
  \State {Seperate MintueData with commas and put in a list.}
  \State {Call a list S.}
  \State {Append S in float and save it.}\\
  \Return {S = Stock prices}
\end{algorithmic}
\end{algorithm}
\subsubsection{Yahoo Finance Stock Prices}
\item  Get Stock Prices : This function obtains the stock prices which are downloaded from yahoo finance and stored
in csv file by filename. It will provide the yahoo API based minute to minute data.
\begin{algorithm}
\caption{Get Stock Prices}
\bigskip
\textbf{INPUT}: Filename\\
\textbf{OUTPUT}: S (Stock Prices) from CSV file.
\begin{algorithmic}[1]
  \item Open Yahoo Finace link.
  \item Open and read the csv Filename as universal.
  \item Store as Reader
  \item Skip the header.
  \item Create a empty list.
  \For {Put row in Reader}
  \If {the second row is numberical string}
  \State {Add second row to empty list}
  \item Call it S.\\
  \Return{S}
\end{algorithmic}
\end{algorithm}
\newpage
\subsubsection{ Choosing either Google Ticker Or Yahoo CSV file}
\item  Class Constructor: This function will analyzie wheather csv filename for Yahoo or Ticker 
for Google finance is used.
\begin{algorithm}
\caption{--init--}
\bigskip
\textbf{INPUT}: User Keyword  \\
\textbf{OUTPUT}: S 
\begin{algorithmic}[1]
  \If {Filename is in keyward Usage}
  \State {Obtain Stock Prices from Get Stock Prices.}
  \Elif {Ticker is used in keyward Usage}
  \State {Obtain Stock Prices from Get Google Data.}
  \Else { Bad parameters}\\
  \Return{S}
\end{algorithmic}
\end{algorithm}
\subsection {Example}
\newpage
\section{Floren Zmirou Class}
This implements equation (\ref{florenZmirouEquation})
\paragraph{Process to solve Floren Zmirou}
Now we have Stock Prices from Stock class. We will obtain list of sigma values from list of Stock Prices.We will explain this class in following algorithms:\newline
T = 60*n where T is the minute to minute time period [0,T].\\
Now we will derive hn, x-step\_size, and x values which will be used in Floren Zmirou Estimator.
  (Derive_hn): Input for this function is Stock Prices which we obtained from Stock class.
  This function will derive hn which will be used in Floren Zmirou Estimator.
  \begin{algorithm}
  \caption{Derive hn}
  \bigskip
  \textbf{INPUT}: Stock Prices S\\
  \textbf{OUTPUT}: hn
  \begin{algorithmic}[1]
  \item n = length of Stock Prices S.
  \item hn = 1.0/n**(1.0/3.0)
  \item return hn
  \end{algorithmic}
  \end{algorithm}
  (x_step_size): Input for this function is Stock Prices which we obtained from Stock class.
  This function will derive x\_step\_size which will be used  to create step size to generate x.
  \begin{algorithm}
  \caption{x\_hn = x step size}
  \bigskip
  \textbf{INPUT}: Stock Prices S\\
  \textbf{OUTPUT}: $\frac{x}{hn}$
  \begin{algorithmic}[1]
  \item We obtained hn from Derive hn function.
  \item Double hn = 2 **hn
  \item Difference = max S-min S
  \item x\_hn = Difference * Double hn
  \item Return x\_hn
  \end{algorithmic}
  \end{algorithm}
  (Derive_x_values): Input for this function is Stock Prices which we obtained from Stock class.
  This function will derive x\_values which will be used  in Floren Zmirou Estimator.
  \begin{algorithm}
  \caption{Derive x Values}
  \bigskip
  \textbf{INPUT}: Stock Prices S\\
  \textbf{OUTPUT}: x
  \begin{algorithmic}[1]
  \item halfh\_n = x\_hn /2.0
  \item x = empty list.
  \item Append x with min S + halfh\_n
  \item We initilaize ex to be first element of x's list.
  \While {ex < max S}
  \State{ex = ex+x\_hn}
  \EndWhile
  \item We will append ex into x's list
  \item Return x
  \end{algorithmic}
  \end{algorithm}
Now we have all the ingredients which will be usefull to solve floren zmirou estimator.
We will implement floren Zmirou on the following functions. Floren Zmirou has Sublocal time, Local time, volatility Estimator and Indicator function.
(Sublocal Time): Input for this function are T,S,x,n,hn
Output for this function will be $L_{T}^n(x) = (\frac{T}{2nhn}) \sum_{i =1}^ n 1_{\left\vert S_{t_i} - x \right\vert < h_n}$
  \begin{algorithm}
  \caption{Sublocal Time}
  \bigskip
  \textbf{INPUT}: T=Time,S= Stock Prices,x= grid points,n= number of total Stock Prices,hn=Step Size\\
  \textbf{OUTPUT}: $L_{T}^n(x) = (\frac{T}{2nhn}) \sum_{i =1}^ n 1_{\left\vert S_{t_i} - x\right\vert }$
  \begin{algorithmic}[1]
  \item sum = 0.0
  \item scalar = T/(2.0*n*hn)
  \For {i in range of the length of Stock Prices}
  \State{We initialize Sti to be the ith element of the list of Stock prices}
  \State{absoluteValue = abs(Sti-x)}
  \State{ We initilize indicatorValue to pass through indicator function}
  \If{absoluteValue is less than hn}
  \State{sum = sum+indicatorValue}
  \Return{scalar*sum}
  \end{algorithmic}
  \end{algorithm}
(Local Time): Input for this function are T,S,x,n,hn.
Output for this function will be $L_{T}^n(x) = (\frac{T}{2nh_n}) \sum_{i =1}^ n 1_{\left\vert S_{t_i} - x\right\vert  < hn}*n(S(t(i+1))-S(t(i))^2$
  \begin{algorithm}
  \caption{Local Time}
  \bigskip
  \textbf{Inputs}: T=Time,S= Stock Prices,x= grid points,n= number of total Stock Prices,hn=Step Size\\
  \textbf{Steps}
  \begin{algorithmic}[1]
  \item sum = 0.0
  \item scalar = T/(2.0*n*hn)
  \For{for i in range of the length of Stock prices - 1}
  \State{We initialize Sti to be the ith element of the list of Stock prices}
  \State{We initialize Stj to be the jth element of the list of Stock prices}
  \State{absoluteValue = abs(Sti-x)}
  \State{Difference = (Stj-Sti)**2}
  \State{ We initilize indicatorValue to pass through indicator function}
  \If{ absoluteValue is less than hn}
  \State{sum = sum+indicatorValue*n*Difference}
  \Return{scalar*sum}
  \end{algorithmic}
  \end{algorithm}
(Volatility Estimator Sn(x)): Volatility Estimator is $\sigma^2(x)$.Input for this function are T,S,x,n,hn.
Output for this function will be Local Time$/$Sublocal Time.
\begin{algorithm}
\caption{Volatility Estimator}
\bigskip
\textbf{Inputs}: T=Time,S= Stock Prices,x= grid points,n= number of total Stock Prices,hn=Step Size\\
\textbf{Steps}
\begin{algorithmic}[1]
\item We will initialize ratio to be Local Time / Sublocal Time.
\item Return ratio.
\end{algorithmic}
\end{algorithm}
We have $\sigma^2(x)$ for Stock Prices. Now we want to see how many stock prices are in each grid point.
If there are less than 0 or 1 percent of stock prices in grid then we will exclude that grid point from
our calculation process.
(DoGridAnalysis):Input for this function are T,S,x,n,hn.
Output for this function give us the list of usable grid points and for each usable grid point, the list of Stock prices.
  \begin{algorithm}
  \caption{DoGridAnalysis}
  \bigskip
  \textbf{INPUT}: T=Time,S= Stock Prices,x= grid points,n= number of total Stock Prices,hn=Step Size,Y = Y percent of total data points.\\
  \textbf{OUTPUT}:
  \begin{algorithmic}[1]
  \item x = useableGridPoints.
  \item d = empty dictionary
  \item Si =  length of stock prices.
  \For{grid Points in x}
  \State{create a dictionary where the keys are the grid points and the value is a list for each grid point}
  \For{stockPrice in S}
  \If{|grid Points - stockPriceCount|$\leq$ hn}
  \State{Add x value to corresponding Si}
  \For{grid Points in x}
  \If{if the list of data points corresponding to x values than Y percent of total grid points}
  \State{add it to the list of usable grid points}
  \State L = dictionary with key values of grid points.
  \State N = Length of L in float.
  \State {Percent of Stock Prices = N/S}
  \If {Percent of Stock prices < Y}
  \State{Remove grid points from x}
  \State{Delete L}\\
  \Return{x = usable grid points and d }
  \end{algorithmic}
  \end{algorithm}
(GetGridVariance:Input for this function is self.
  \begin{algorithm}
  \caption{GetGridVariance}
  \bigskip
  \textbf{INPUT}: \\
  \textbf{OUTPUT}: Standard Deviation of usable grid points
  \begin{algorithmic}[1]
  \item Points = Empty list
  \For{x in usable grid points}
  \State{y =Sn(x))}
  \State{Put (x,y) in Points's list}\\
  \Return{Points}
  \end{algorithmic}
  \end{algorithm}
(GetGridInverseStandardDeviation):Input for this function is self.
  \begin{algorithm}
  \caption{GetGridInverseStandardDeviation}
  \bigskip
  \textbf{INPUT}: \\
  \textbf{OUTPUT}: Standard Deviation of usable grid points
  \begin{algorithmic}[1]
  \item Points = Empty list
  \For{x in usable grid points}
  \State{y = 1/sqrt(Sn(x))}
  \State{Put (x,y) in Points's list}\\
  \Return{Points}
  \end{algorithmic}
  \end{algorithm}\\
{\bf Natural Cubic Spline}- To construct the cubic spline interpolant $S$ for the function $f$, defined
at the numbers $x_0<x_1<\ldots<x_n$ satisfying $S''(x_0) = S''(x_n) = 0:$\\
(Natural Cubic Spline)
  \begin{algorithm}
  \caption{Natural Cubic Spline}
  \bigskip
  \textbf{INPUT}: $n, x_0,x_1,\ldots,x_n; a_0 = f(x_0), a_1 = f(x1),\ldots, a_n = f(x_n)$\\
  \textbf{OUTPUT}: $a_j, b_j,c_j,d_j$ for $j = 0,1,\ldots,n-1$.\\
  (note: $S_(x) = S_j(x) = a_j+b_j(x-x_j)+c_j(x-x_j)^2+d_j(x-x_j)^3$ for $x_j\leq x \leq x_{j+1}$)
  \begin{algorithmic}[1]
  \For{$i = 0,1,$\ldots$,n-1$}
  \State{$h_i = x_{i+1}-x_i$}
  \EndFor
  \For{$i = 1,2,\ldots,n-1$}:
  \State{$\alpha_i = \frac{3}{h_i}(a_{i+1}-a_i)-\frac{3}{h_{i-1}}(a_i-a_{i-1})$}
  \EndFor
  \item Set $l_0 = 1;$
  \item $\mu = 0;$
  \item $z_0 = 0.$
  \For{$i = 1,2,\ldots,n-1$}
  \State{$l_1 = 2(x_{i+1}-x_{i-1})-h_{i-1}\mu_{i-1}$};
  \item $\mu_1 = \frac{h_1}{l_1};$
  \item $z_1 = \frac{(\alpha_1 - h_{i-1}z_{i-1})}{l_1}.$
  \item Set $l_n = 1;$
  \item $z_n = 0;$
  \item $c_n = 0.$
  \For{$j = n-1,n-2,\ldots,0$}
  \State{$ c_j = z_j-\mu_jc_{j+1};$}
  \State{$b_j = (a_{j+1}-a_j)/h_j -h_j(c_{j+1}+2c_j)/3;$}
  \State{$d_j = (c_{j+1}-c_j)/(3h_j)$}
  \item Return $(a_j,b_j,c_j,d_j for j = 0,1,\ldots,n-1)$;
  \end{algorithmic}
  \end{algorithm}
\end{enumerate}
